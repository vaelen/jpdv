\documentclass[11pt,landscape]{article}

\usepackage[utf8]{inputenc}

\usepackage{CJK}
\usepackage{fullpage}

\usepackage{color}

\usepackage{tikz}

\newcommand{\jptext}[1]{\begin{CJK}{UTF8}{min}#1\end{CJK}}

\begin{document}

\begin{center}

\begin{tikzpicture}[sibling distance=40mm,level distance=10mm]
\tikzstyle{every node}=[draw]
\node {S} child { node {\jptext{\textcolor{red}{日本語}。}} child { node[blue] {\jptext{は}} child { node {\jptext{\textcolor{red}{これ}}}  } } } ;
\end{tikzpicture}
\hspace{2em}
\begin{tikzpicture}[sibling distance=40mm,level distance=10mm]
\tikzstyle{every node}=[draw]
\node {S} child { node {\jptext{表示。}} child { node[blue] {\jptext{を}} child { node {\jptext{\textcolor{red}{候補}}} child { node[blue] {\jptext{の}} child { node {\jptext{\textcolor{red}{キーワード}}}  } } } } child { node[blue] {\jptext{に}} child { node {\jptext{検索\textcolor{red}{ボックス}}}  } } } ;
\end{tikzpicture}

\vspace{2em}

\begin{tikzpicture}[sibling distance=40mm,level distance=10mm]
\tikzstyle{every node}=[draw]
\node {S} child { node {\jptext{\textcolor{red}{連れ}て行きたい。}} child { node[blue] {\jptext{を}} child { node {\jptext{\textcolor{red}{娘}}} child { node[blue] {\jptext{と}} child { node {\jptext{\textcolor{red}{妻}}}  } } } } child { node[blue] {\jptext{へ}} child { node {\jptext{\textcolor{red}{日本}}}  } } child { node[blue] {\jptext{は}} child { node {\jptext{ヤング\textcolor{red}{さん}}}  } } } ;
\end{tikzpicture}

\end{center}


\end{document}

%%% Local Variables:
%%% coding: utf-8
%%% mode: latex
%%% End: